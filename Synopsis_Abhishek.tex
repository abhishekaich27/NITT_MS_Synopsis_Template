\documentclass[12pt,a4paper]{article}
\usepackage{enumerate, amsmath, amsthm, multirow, mathrsfs, array, amssymb, indentfirst, latexsym, pstricks, graphicx, epstopdf, subfig, float, comment, bm, xpatch}
\usepackage{times}
\usepackage[labelfont=bf,textfont=bf]{caption}
\captionsetup[figure]{labelfont={bf},name={Fig.},labelsep=period}
\allowdisplaybreaks[1]
\newcommand{\halmos}{\rule{2mm}{2.5mm}}
\renewcommand{\qedsymbol}{\halmos}
\theoremstyle{plain}
\newtheorem{thm}{Theorem}[section]
\newtheorem{lem}[thm]{Lemma}
\newtheorem{cor}[thm]{Corollary}
\theoremstyle{definition}
\newtheorem{dfn}[thm]{Definition}
\newtheorem{exm}[thm]{Example}
\topmargin=-.6in
\textheight=9.6in
\oddsidemargin=5pt
\textwidth=38.5pc
\linespread{1.3}
\usepackage{hyperref}
\usepackage[round]{natbib}
\begin{document}

\begin{titlepage}

\thispagestyle{empty}
\begin{center}
\thispagestyle{empty}
\normalsize
\textsc{\textbf{SYNOPSIS OF}}\\
\bigskip

\Large{\textbf{Enter title here in Uppercase}}\\

\bigskip \bigskip \bigskip \bigskip \bigskip \bigskip \bigskip \bigskip


\normalsize
\textbf{A THESIS}\\
\emph{to be submitted by}\\
\bigskip \bigskip \bigskip \bigskip
\normalsize
\textbf{Name in Uppercase}\\
\bigskip \bigskip \bigskip \bigskip
\emph{for the award of the degree}\\ \bigskip \bigskip
\emph{of}\\
\bigskip \bigskip
\normalsize
\textbf{MASTER OF SCIENCE (BY RESEARCH)}\\
\bigskip \bigskip \bigskip \bigskip \bigskip \bigskip \bigskip \bigskip
\centering

\includegraphics[width=32mm, height=32mm]{NITT.jpg}

\bigskip \bigskip
\normalsize
\textbf{Department in Uppercase} \\
\normalsize
\textbf{
NATIONAL INSTITUTE OF TECHNOLOGY\\
TIRUCHIRAPPALLI -- 620 015}\\ \bigskip
\normalsize
\textbf{Month (Uppercase) year}

\end{center}
\end{titlepage}

%Instructions:
% The synopsis should not cross more than 20 pages. 

% All the formats (captions, section, subsection heading format, table caption format) are set according to the template given by the Institute. Do not change anything except the content.

% Citation style has been set according to the format. Just update the reference.bib file and it will be set. Do not change anything else regarding reference. Use this syntax:[\citet{.}]

% Suggestions:
% 1. Add your figures in the 'figs' folder depository. This helps in keeping the interface clean.

% 2. Create label for each equation, figure, table, may be even sections also. Then use (\ref{.}) command for equations and \ref{.} for figures, tables and sections. Doing it this way, you won't be worried about the equation numbers or figure, table numbers in the text. It will get updated automatically whenever you add or remove new equations, figures and tables. 

% 3. Don't add more than 3-4 pages of references. You can add more in your thesis.
\newpage
\section{INTRODUCTION}% 1 page is enough
Direction of arrival (DOA) estimation refers to the process of retrieving the direction information of several sources from the outputs of an number of receiving sensors that form a array. DOA estimation is a major problem in array signal processing and has wide applications in radar, sonar, wireless communications, etc. The study of DOA estimation methods has a long history. My algorithms like Capon’s beamformer were initially proposed for the estimation of closely spaced sources [\citet{R1}]. Since the 1970s, when Pisarenko found that the DOAs can be retrieved from data's second order statistics [\citet{R2}], a prominent class of methods designated as subspace-based methods have been developed, e.g., the multiple signal classification (MUSIC) and the estimation of parameters by rotational invariant techniques (ESPRIT) along with their variants [\citet{R3,R4,R5,R6,R7}].

Compressive sensing (CS) [\citet{R8,R9}] is being widely used in different research areas for its various properties. It utilizes the fact that the signals impinging on a sensor array are spatially sparse thus letting to exploit CS theory in DOA estimation problem. The novel contribution of CS-based DOA estimation methods is that very less time samples are required for high-resolution performance even in highly noisy environment. Recent works in DOA estimation have the introduction of CS to the framework. These new methods are motivated by techniques in sparse representation and compressed sensing methodology [\citet{R10,R11,R12}], and most of them have been proposed during the last decade. The sparse estimation (or optimization) methods can be applied in several demanding scenarios, including cases with no knowledge of the source number, limited number of time samples (even a single time sample), and highly or completely correlated sources. Due to these attractive properties they have been extensively studied and their popularity is reflected by the large number of publications.

It is important to note that there is a key difference between the common sparse representation framework and DOA estimation framework. To be specific, the studies of sparse representation have been focused on discrete linear systems. In contrast to this, the DOA parameters are continuous valued and the observed data are nonlinear in the angle domain. Depending on the model adopted, we can classify the sparse methods for DOA estimation into three categories, namely, on-grid, off-grid and gridless. For on-grid sparse methods, the data model is obtained by assuming that the true DOAs lie on a set of fixed grid points in order to straightforwardly apply the existing sparse representation techniques. While a grid is still required by off-grid sparse methods, the DOAs are not restricted to be on the grid. Finally, the recent gridless sparse methods do not need a grid, as their name suggests, and they operate directly in the continuous domain.

% Motivation
\section{MOTIVATION}
While DOA estimation is now a mature field with a solid theoretical basis and a large number of practical applications, it is still an evolving and quite active field of research. The major class of algorithms, the subspace based methods, suffer from certain well-known limitations. For example, they need a priori knowledge on the source number that may be difficult to obtain. Additionally, method's like Capon’s beamformer, MUSIC and ESPRIT are covariance-based and require a sufficient number of data snapshots to accurately estimate the data covariance matrix. Moreover, they can be sensitive to source correlations that tend to cause a rank deficiency in the sample data covariance matrix.

The main purpose of the DOA algorithm is to estimate the direction of incoming signals based on samples of received signals. The accuracy of estimation depends on the number of received signal samples. In general, DOA algorithms depend on two fundamental descriptions: steering vector and covariance matrix. Steering vector represents the response of the antenna array to a source in a certain direction. For multiple signals impinging an array, each signal direction will be associated with a unique steering vector. On top of that, the steering vector also describes the array geometry since different array configuration leads to unique expression. The covariance matrix involves the expectation of a received signal, but in practice this cannot be obtained; therefore, an average over a finite sample of received signals is normally used instead [\citet{R13}].

In many of the emerging applications, the traditional Nyquist rate of data acquisition is used but it is so high that we end up with far too many samples. It may simply be too costly, or even physically impossible, to build devices capable of acquiring samples at the required rate. Thus, despite tremendous advances in handling computational power, acquisition and processing of signals in areas such as medical imaging, remote surveillance, radar systems and data analysis continues to pose a tremendous challenge.

Compressive Sensing is a method of undersampling which aims at finding the most sparse representation of a signal that is able to achieve a target level with acceptable distortion. The CS beamformer proposed in [\citet{R14}] modifies the data model of the DOA estimation problem in terms of a CS problem and then solves it with traditional estimation based methods such as MVDR, MUSIC algorithms. This compensates for the price of the high complexity these algorithms need for the estimation of DOAs.

% Literature Survey 
\section{LITERATURE SURVEY}
One way to tackle the heavy computational complexity in subspace based algorithms is to introduce CS based Beamformer (CSB). Not only CSB reduces the data required, it also reduces the over flops required drastically [\citet{R14}]. 

CS sparse signal reconstruction algorithms can be broadly classified into two categories – Convex optimization based algorithms and the class of greedy algorithms. In this thesis, we mainly concentrate on a greedy algorithm called as orthogonal matching pursuit (OMP) [\citet{R15}]. The typical greedy sparse recovery approaches are basis pursuit (BP), compressive sampling matching pursuit (CoSaMP) [\citet{R16}], sparse Bayesian learning (SBL) and OMP [\citet{R17}]. Among these approaches, OMP is a compelling and favorable candidate for solving the DOA estimation problem due to its attractive properties. The main advantages of this algorithm are its low computational complexity and speed of recovery. Also, in addition to its fast implementation, OMP is also empirically advantageous in terms of recovery performance [\citet{R18}].

A literature survey reveals some of the most interesting applications of CS to DOA estimation problem in recent studies. Sparse recovery algorithms are equivalently effective with minimal snapshots, unlike as with multiple snapshots in case of conventional high-resolution DOA estimation algorithms. Hence, it is easier to work in scenarios like dynamic object tracking. The following work on application of CS to DOA estimation is a testimony to the advantages over traditional methods. In [\citet{R19}], author's have given an overview of these sparse methods for DOA estimation particularly recently developed gridless sparse methods. In [\citet{R20,R21}], a sparse covariance-based representation is exploited for source localization by applying a global matched filter. In [\citet{R22}], the $\ell_1$-SVD sparse recovery algorithm is proposed for DOA estimation. In [\citet{R23,R24}], an iterative re-weighted $\ell_1$ minimization is proposed for source estimation. In [\citet{R25}], a mixed $\ell_{2,0}$-norm based joint sparse approximation technique is proposed (JLZA-DOA) to solve the DOA estimation problem. Algorithms in [\citet{R26,R27,R28}] address the DOA estimation problem by directly representing the array output in time domain with an over-complete basis from the array response vector.
%

\section{OBJECTIVES AND SCOPE OF WORK}
\label{binfib}
This thesis focuses on utilizing an inherent property of Uniform Linear Array (ULA) based observed data, called sparsity. The main goal is to improve the complexity computations and the size of input data required to detect the directions of sources. The primary scope of this thesis is briefly summarized below:

\begin{itemize}
\item The first goal of this thesis is to derive stricter bounds for the dimensions of the measurement matrix employed in CSB-MUSIC algorithm.

\item The second goal of this thesis is to propose a novel and efficient CS beamformer root-MUSIC algorithm which follows the strict bounds introduced above. With this CSB introduction, the subspace deviation is analyzed and derived for this new algorithm in low snapshot scenario.

\item The third goal of this thesis is to overcome the inconsistency in power spectrum of OMP algorithm when employed for DOA estimation. In conjunction to this, a framework is to be designed for source distances estimation.

\end{itemize}
%\newpage
\section{SUMMARY OF THE RESEARCH WORK}

%Research work #1
\subsection{Application of Compressive Sensing Based Beamformer to Music Algorithm for DOA Estimation}
% keep each subsection dedicated to one research paper
% 1-1.5 page each per section
\textbf{\emph{Background}}: 
% 8-9 lines
\subsubsection{CS Based Data Model}

\textbf{\emph{Results and discussion}}: 
% Brief discussion including 1-2 figures

%Research work #2 (Follow same instructions as above)
\subsection{CSB - root - MUSIC Algorithm}
\textbf{\emph{Background}}: 

\subsubsection{Proposed Algorithm}

\textbf{\emph{Results and discussion}}: 

% Create as many subsections as you want. Preferably #Research papers = #subsections

\section{SUMMARY AND CONCLUSIONS}
In this thesis, sparse representation based system model is exploited for DOA estimation. In particular, the following works are done:

\begin{itemize}
% each item point (3-5 lines) for each subsection you created

\item
% Conclusion of Research work 1

\item  

% Conclusion of Research work 2

\end{itemize}

% Referenaces
% NOTE: Do not change anything in the next three lines
\clearpage
\bibliographystyle{mystyle} 
\bibliography{references.bib}

% Update the contents according to your requirements. Do not change the style.
\newpage
\noindent \textbf{\normalsize 
\begin{center}
PROPOSED CONTENTS OF THE THESIS
\end{center}
} 
\vspace*{0.15cm}
\begin{description}
	\item[CHAPTER 1]  \textbf{INTRODUCTION}
	\begin{description}
     
	\item[1.1] ULA System Data Model
	\item[1.2] Compressive Sensing
	\item[1.3] The Problem and Motivation
    \item[1.4] Contributions of the Work
    \item[1.5] Organization of the Thesis
\end{description}

	\item[CHAPTER 2] \textbf{LITERATURE REVIEW}
\begin{description}

	\item[2.1] Subspace Based DOA Estimation Algorithms  
	\item[2.2] Compressive Sensing Based Sparse Recovery Algorithms  
	\item[2.3] Sparse Framework Based DOA Estimation Algorithms
   \end{description}
   
\item[CHAPTER 3] \textbf{COMPRESSIVE SENSING BASED MUSIC ALGORITHM}	
\begin{description}
    \item[3.1] Introduction
    \item[3.2] Preliminaries
    \item[3.3] CS Beamformer MUSIC Algorithm 
		\begin{description}
		\item[3.3.1] Choosing a Proper Measurement Matrix $\Phi$
        \item[3.3.2] Proposed Bound to the Dimension of $\Phi$
		\end{description}
	\item[3.4] Results and Discussion
\end{description}

\item[CHAPTER 4]\textbf{COMPRESSIVE SENSING BASED ROOT - MUSIC ALGORITHM}
\begin{description}
	\item[4.1] Introduction
	\item[4.2] Proposed Algorithm 
		\begin{description}
		\item[4.2.1] CS beamformer root-MUSIC Algorithm
        \item[4.2.2] Remarks on \textit{m} = \textit{K} + 1
		\end{description}
	\item[4.3] Subspace Deviation in CS Beamformer 
    \item[4.4] Results and Discussion\\*\\
\end{description}

\item[CHAPTER 5]  \textbf{OVERCOMING DOA SPECTRUM INCONSISTENCY OF ORTHOGONAL MATCHING PURSUIT ALGORITHM} 
\begin{description}
  \item[5.1] Introduction
  \item[5.2] Preliminaries
  		\begin{description}
		\item[5.2.1] CS Data Model for DOA Estimation
        \item[5.2.2] Application of OMP Algorithm
		\end{description}
  \item[5.3] Proposed Algorithm 
		\begin{description}
		\item[5.3.1] On the Bound of Weak-1 RIP Constant
        \item[5.3.2] Distance of Sources
		\end{description}
  \item[5.4] Results and Discussion
\end{description}

\item[CHAPTER 6] \textbf{SUMMARY AND CONCLUSION}
\begin{description}
	\item[6.1] Summary and Conclusion
	\item[6.2] Future Work
\end{description}

\end{description}

% List all your research papers as below as in given format
\newpage
\begin{center}
\textbf{THE THESIS IS BASED ON THE FOLLOWING PUBLICATIONS}
\end{center}
\bigskip

\begin{itemize}

\item[I] \textbf{PRESENTATIONS IN CONFERENCES}

\begin{enumerate}[ 1. ]
\item \textbf{Aich, A.} and \textbf{P. Palanisamy} (2016) A strict bound for dimension of measurement matrix for CS beamformer MUSIC algorithm. \emph{IEEE Region 10 Conference (TENCON-2016)}, NTU Singapore, 2602-2605.

\item \textbf{Aich, A.} and \textbf{P. Palanisamy} (2017) A novel CS beamformer root-MUSIC algorithm and its subspace deviation analysis. \emph{IEEE Region 10 Conference (IEEE TENCON-2017)}, Penang, Malaysia, 1404-1408.

\end{enumerate}

\item[II] \textbf{REFERRED JOURNALS}
\begin{enumerate}[ 1. ]
\item \textbf{Aich, A., P. Palanisamy,} and \textbf{N. Kalyanasundaram} A Novel Sparse recovery based DOA estimation algorithm by relaxing the RIP constraint. \emph{IET-Signal Processing (Revision submitted)}.

\end{enumerate}
\end{itemize}

 \end{document}